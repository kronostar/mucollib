\documentclass[12pt,a4paper,final,twoside,titlepage]{book}
\usepackage[utf8]{inputenc}
\usepackage{amsmath}
\usepackage{amsfonts}
\usepackage{amssymb}
\usepackage{makeidx}
\usepackage{graphicx}
\usepackage{longtable}
\author{\LARGE Steve Martin}
\title{\Huge MuColLib}
\usepackage{hyperref}
\hypersetup{colorlinks, 
           citecolor=black,
           filecolor=black,
           linkcolor=black,
           urlcolor=black,
           bookmarksopen=true}
\begin{document}
\maketitle
\null
\vfill
Copyright \copyright{}  2018,2019  Steve Martin.
Permission is granted to copy, distribute and/or modify this document
under the terms of the GNU Free Documentation License, Version 1.3
or any later version published by the Free Software Foundation;
with no Invariant Sections, no Front-Cover Texts, and no Back-Cover Texts.
A copy of the license is included in the section entitled ``GNU
Free Documentation License''.
\frontmatter
\tableofcontents
\newpage
\chapter{Introduction}
This is the manual for \textbf{\textit{MuColLib}}, the \textbf{Mu}sic \textbf{Col}lection \textbf{Lib}rary tool. A database geared at providing a comprehensive resource of information about a music collection.  This goes way beyond just recording the Artist and Album title, although it can do that too!
\\

The list of tables presented in Part II, represent the initial basis upon which this tool is built: there may be changes in the future.
 
\mainmatter
\part{Guide}
\chapter{Getting Started}
\section{Installation}
At this moment in time installation is pretty basic, we have not packaged the program in any way.  So, for now, the best way to install is to download the programs, install them somewhere appropriate, and run them with Python 3.7.
\pagebreak 
\section{First run}
When the program is run for the first time it will create a database, because one does not exist, and will present the following screen:

\includegraphics[scale=1]{Images/firstRun.png}

For now we will assume that you do not have an available CSV file in the correct format (more on this below), and you select \textbf{\textit{No}}

\section{Start screen}
Once past the first run screen, or if this is not the first time you have run the program, you will see this screen:

\includegraphics[scale=1]{Images/firstScreen.png}

Which, of course, would ordinarily be populated with elements of your music library.

\chapter{Adding Entries}
\section{Artists}
\section{Albums}\phantomsection
\part{Database}
\chapter*{Tables}
\phantomsection
\addcontentsline{toc}{chapter}{Tables}
\renewcommand\thetable{1.\arabic{table}}
\setcounter{table}{0}

%
% Structure: Album
%
\begin{longtable}{|c|c|c|c|c|} 
\caption{Structure of table Album} 
\label{tab:Album-structure} \\
\hline 
\multicolumn{1}{|c|}{\textbf{Column}} & \multicolumn{1}{|c|}{\textbf{Type}} & \multicolumn{1}{|c|}{\textbf{Null}} & \multicolumn{1}{|c|}{\textbf{Default}} & \multicolumn{1}{|c|}{\textbf{Links to}} \\
\hline
\textbf{\textit{id}} & int(11) & No &  &  \\ 
\hline 
Name & varchar(120) & No &  &  \\ 
\hline 
Year & year(4) & No &  1900 &  \\ 
\hline 
OrigYear & year(4) & No & 1900 &  \\ 
\hline 
ArtistId & int(11) & No &  & Artist (id) \\ 
\hline 
FormatId & int(11) & No &  & Format (id) \\ 
\hline 
GenreId & int(11) & No &  & Genre (id) \\ 
\hline 
LabelId & int(11) & No &  & Label (id) \\ 
\hline 
OrigLabelId & int(11) & No &  & Label (id) \\ 
\hline 
\end{longtable}

%
% Structure: AlbumMusician
%
%\begin{longtable}{|c|c|c|c|c|} 
%\caption{Structure of table AlbumMusician} 
%\label{tab:AlbumMusician-structure} \\
%\hline 
%\multicolumn{1}{|c|}{\textbf{Column}} & \multicolumn{1}{|c|}{\textbf{Type}} & \multicolumn{1}{|c|}{\textbf{Null}} & \multicolumn{1}{|c|}{\textbf{Default}} & \multicolumn{1}{|c|}{\textbf{Links to}} \\ 
%\hline
%\textbf{\textit{id}} & int(11) & No &  &  \\ 
%\hline 
%Musician\_id & int(11) & No &  & Musician (id) \\ 
%\hline 
%Album\_id & int(11) & No &  & Album (id) \\ 
%\hline 
%MusicianStatus\_id & int(11) & No &  & MusicianStatus (id) \\ 
%\hline 
%\end{longtable}

%
% Structure: Artist
%
\begin{longtable}{|c|c|c|c|} 
\caption{Structure of table Artist} 
\label{tab:Artist-structure} \\
\hline 
\multicolumn{1}{|c|}{\textbf{Column}} & \multicolumn{1}{|c|}{\textbf{Type}} & \multicolumn{1}{|c|}{\textbf{Null}} & \multicolumn{1}{|c|}{\textbf{Default}} \\ 
\hline
\textbf{\textit{id}} & int(11) & No &  \\ 
\hline 
Name & varchar(250) & No &  \\ 
\hline 
Sort & varchar(250) & No &  \\ 
\hline 
\end{longtable}

%
% Structure: Format
%
\begin{longtable}{|c|c|c|c|} 
\caption{Structure of table Format} 
\label{tab:Format-structure} \\
\hline 
\multicolumn{1}{|c|}{\textbf{Column}} & \multicolumn{1}{|c|}{\textbf{Type}} & \multicolumn{1}{|c|}{\textbf{Null}} & \multicolumn{1}{|c|}{\textbf{Default}} \\ 
\hline
\textbf{\textit{id}} & int(11) & No &  \\ 
\hline 
Name & varchar(20) & No &  \\ 
\hline 
\end{longtable}

%
% Structure: Genre
%
\begin{longtable}{|c|c|c|c|} 
\caption{Structure of table Genre} 
\label{tab:Genre-structure} \\
\hline 
\multicolumn{1}{|c|}{\textbf{Column}} & \multicolumn{1}{|c|}{\textbf{Type}} & \multicolumn{1}{|c|}{\textbf{Null}} & \multicolumn{1}{|c|}{\textbf{Default}} \\ 
\hline
\textbf{\textit{id}} & int(11) & No &  \\ 
\hline 
Name & varchar(45) & No &  \\ 
\hline 
\end{longtable}

%
% Structure: Instrument
%
%\begin{longtable}{|c|c|c|c|} 
%\caption{Structure of table Instrument} 
%\label{tab:Instrument-structure} \\
%\hline 
%\multicolumn{1}{|c|}{\textbf{Column}} & \multicolumn{1}{|c|}{\textbf{Type}} & \multicolumn{1}{|c|}{\textbf{Null}} & \multicolumn{1}{|c|}{\textbf{Default}} \\ 
%\hline
%\textbf{\textit{id}} & int(11) & No &  \\ 
%\hline 
%Name & varchar(20) & No &  \\ 
%\hline 
%\end{longtable}

%
% Structure: Label
%
\begin{longtable}{|c|c|c|c|} 
\caption{Structure of table Label} 
\label{tab:Label-structure} \\
\hline 
\multicolumn{1}{|c|}{\textbf{Column}} & \multicolumn{1}{|c|}{\textbf{Type}} & \multicolumn{1}{|c|}{\textbf{Null}} & \multicolumn{1}{|c|}{\textbf{Default}} \\ 
\hline
\textbf{\textit{id}} & int(11) & No &  \\ 
\hline 
Name & varchar(45) & No &  \\ 
\hline 
\end{longtable}

%
% Structure: Musician
%
%\begin{longtable}{|c|c|c|c|c|} 
%\caption{Structure of table Musician} 
%\label{tab:Musician-structure} \\
%\hline 
%\multicolumn{1}{|c|}{\textbf{Column}} & \multicolumn{1}{|c|}{\textbf{Type}} & \multicolumn{1}{|c|}{\textbf{Null}} & \multicolumn{1}{|c|}{\textbf{Default}} & \multicolumn{1}{|c|}{\textbf{Links to}} \\ 
%\hline
%\textbf{\textit{id}} & int(11) & No &  &  \\ 
%\hline 
%Artist\_id & int(11) & No &  & Artist (id) \\ 
%\hline 
%Instrument\_id & int(11) & No &  & Instrument (id) \\ 
%\hline 
%\end{longtable}

%
% Structure: MusicianStatus
%
%\begin{longtable}{|c|c|c|c|} 
%\caption{Structure of table MusicianStatus} 
%\label{tab:MusicianStatus-structure} \\
%\hline 
%\multicolumn{1}{|c|}{\textbf{Column}} & \multicolumn{1}{|c|}{\textbf{Type}} & \multicolumn{1}{|c|}{\textbf{Null}} & \multicolumn{1}{|c|}{\textbf{Default}} \\ 
%\hline
%\textbf{\textit{id}} & int(11) & No &  \\ 
%\hline 
%Name & varchar(20) & No &  \\ 
%\hline 
%\end{longtable}

%
% Structure: Song
%
%\begin{longtable}{|c|c|c|c|c|} 
%\caption{Structure of table Song} 
%\label{tab:Song-structure} \\
%\hline 
%\multicolumn{1}{|c|}{\textbf{Column}} & \multicolumn{1}{|c|}{\textbf{Type}} & \multicolumn{1}{|c|}{\textbf{Null}} & \multicolumn{1}{|c|}{\textbf{Default}} & \multicolumn{1}{|c|}{\textbf{Links to}} \\ 
%\hline
%\textbf{\textit{id}} & int(11) & No &  &  \\ 
%\hline 
%Name & varchar(120) & No &  &  \\ 
%\hline 
%Length & time & Yes & NULL &  \\ 
%\hline 
%Number & int(11) & No &  &  \\ 
%\hline 
%Album\_id & int(11) & No &  & Album (id) \\ 
%\hline 
%\end{longtable}

%
% Structure: SongComposer
%
%\begin{longtable}{|c|c|c|c|c|} 
%\caption{Structure of table SongComposer} 
%\label{tab:SongComposer-structure} \\
%\hline \multicolumn{1}{|c|}{\textbf{Column}} & \multicolumn{1}{|c|}{\textbf{Type}} & \multicolumn{1}{|c|}{\textbf{Null}} & \multicolumn{1}{|c|}{\textbf{Default}} & \multicolumn{1}{|c|}{\textbf{Links to}} \\ 
%\hline
%\textbf{\textit{id}} & int(11) & No &  &  \\ 
%\hline 
%Artist\_id & int(11) & No &  & Artist (id) \\ 
%\hline 
%Song\_id & int(11) & No &  & Song (id) \\ 
%\hline 
%\end{longtable}

%
% Structure: SongMusician
%
%\begin{longtable}{|c|c|c|c|c|} 
%\caption{Structure of table SongMusician} 
%\label{tab:SongMusician-structure} \\
%\hline \multicolumn{1}{|c|}{\textbf{Column}} & \multicolumn{1}{|c|}{\textbf{Type}} & \multicolumn{1}{|c|}{\textbf{Null}} & \multicolumn{1}{|c|}{\textbf{Default}} & \multicolumn{1}{|c|}{\textbf{Links to}} \\ 
%\hline
%\textbf{\textit{id}} & int(11) & No &  &  \\ 
%\hline 
%Musician\_id & int(11) & No &  & Musician (id) \\ 
%\hline 
%Song\_id & int(11) & No &  & Song (id) \\ 
%\hline 
%MusicianStatus\_id & int(11) & No &  & MusicianStatus (id) \\ 
%\hline 
%\end{longtable}

%
% Structure: SongSection
%
%\begin{longtable}{|c|c|c|c|c|} 
%\caption{Structure of table SongSection} 
%\label{tab:SongSection-structure} \\
%\hline 
%\multicolumn{1}{|c|}{\textbf{Column}} & \multicolumn{1}{|c|}{\textbf{Type}} & \multicolumn{1}{|c|}{\textbf{Null}} & \multicolumn{1}{|c|}{\textbf{Default}} & \multicolumn{1}{|c|}{\textbf{Links to}} \\ 
%\hline
%\textbf{\textit{id}} & int(11) & No &  &  \\ 
%\hline 
%Ref & varchar(10) & No &  &  \\ 
%\hline 
%Name & varchar(120) & No &  &  \\ 
%\hline 
%Song\_id & int(11) & No &  & Song (id) \\ 
%\hline 
%\end{longtable}


\part{Licenses}
\include{fdl}
\newpage
\include{3ClauseBsd}
\end{document}